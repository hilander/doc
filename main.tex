\documentclass[a4paper]{article}
\usepackage{polski}
\usepackage{antpolt}
\usepackage[utf8]{inputenc}
%\usepackage{qswiss,qcourier}
%\usepackage{sfheaders}

%\usepackage[layout]{tools}
\setlength{\voffset}{1cm}
\setlength{\voffset}{1cm}
\author{Mariusz Barycz}
\title{Analiza metod zarządzania wątkami mieszanymi}
\begin{document} 

\maketitle
\thispagestyle{empty}
\newpage
rodzina Computer Modern, często na przykład brakuje w nich kompletu
symboli matematycznych, znaków z alfabetów niełacińskich, takich jak greka
lub alfabet cyrylicki, albo niektórych odmian, jak na przykład kapitalików.

    Z drugiej strony krój cm ma też wady: kreski znaków są cieńsze, a
względna wysokość małych liter2 jest mniejsza niż w wielu innych krojach.
Te cechy kroju cm powodują, że jest mniej czytelny w wypadku, gdy jest
reprodukowany na nośniku o niskiej rozdzielczości, lub – mówiąc wprost –
nie najlepiej się nadaje do dokumentów, które będą wyświetlane na ekranach
komputerów, np. dokumentów w formacie pdf.

\begin{equation}
\sum_{i=0}^{n^2}{x_i} = \frac{1}{b / 2^{k-1}}
\end{equation}
    Pakiet qtimes umożliwia skład dokumentu w kroju QTimes, który jest
klonem znanego kroju Times New Roman autorstwa Stanleya Morisona.
Jeżeli dokument zawiera wzory matematyczne, to aby znaki w formułach
były optycznie zgodne z otaczającym je tekstem, należy także dołączyć
pakiety txfonts oraz qtxmath:
\end{document}
