\documentclass{mwart}
\usepackage{polski}
\usepackage{antpolt}
\usepackage[utf8]{inputenc}
\usepackage{graphicx}
\usepackage[table]{xcolor}
\usepackage[bookmarks, pdfhighlight=/P, linkbordercolor={1 1 1}, unicode=true]{hyperref}
%\usepackage{qswiss,qcourier}
%\usepackage{sfheaders}

%\usepackage[layout]{tools}

%\setlength{\voffset}{1cm}
%\setlength{\voffset}{1cm}
\author{Mariusz Barycz}
\title{Analiza metod zarządzania wątkami mieszanymi}
\date{\today}
\begin{document} 

\maketitle
\thispagestyle{empty}

\newpage
\tableofcontents
\thispagestyle{empty}

\newpage
% Akapity: bez wcięcia, za to z odstępem pionowym.
% małe hackowanie latex-a: właściwości akapitu po spisie treści:)
\setlength{\parindent}{0pt}
\setlength{\parskip}{1ex plus 0.7ex minus 0.1ex}

\begin{center}

\begin{minipage}{10cm}

\setlength{\parindent}{0pt}
\setlength{\parskip}{1ex plus 0.7ex minus 0.1ex}

\begin{center}
\bf{Streszczenie}
\end{center}

\end{minipage}
\end{center}

\newpage
%
\section{Wstęp}
%
\indent 
	Od połowy lat 80. XX~wieku zwiększano moc obliczeniową procesorów za pomocą \emph{równoległego wykonania rozkazów procesora}
	(ang. \emph{instruction-level parallelism, ILP}). Na wykonanie pojedynczego rozkazu składa się wykonanie w~ustalonej kolejności
	określonej liczby kroków. Klasyczna sekwencja składa się z~pięciu kroków:
	\begin{enumerate}
	\item pobranie (ang. \emph{Instruction Fetch, IF});
	\item dekodowanie (ang. \emph{Instruction Decode, ID});
	\item wykonanie (ang. \emph{Execute, EX});
	\item dostęp do pamięci (ang. \emph{Memory Access, MEM});
	\item zapis (ang. \emph{Write Back, WB}).
	\end{enumerate}
	Każdy rozkaz był wykonywany oddzielnie, tj. jednostka centralna wykonywała kroki od IF do WB dla każdego rozkazu osobno, zaś
	podejście zastosowane w ILP polegało na maksymalnym wykorzystaniu części procesora obsługujących poszczególne kroki.
	Jet to \emph{potokowe przetwarzanie rozkazów} (ang. \emph{Pipelining}).
	\begin{center}
	\begin{tabular}{|c|c|c|c|c|c|c|c|c|c|c|} \hline
	& \multicolumn{5}{c|}{Rozkaz 1} & \multicolumn{5}{c|}{Rozkaz 2}  \\ \cline{2-11}
	& IF & ID & \cellcolor{yellow} EX & MEM & WB & IF & ID & EX & MEM & WB\\ \cline{2-11} \cline{2-11}
	Cykl zegara & 1 & 2 & \cellcolor{yellow} 3 & 4 & 5 & 6 & 7 & 8 & 9 & 10 \\ \hline
	\end{tabular}\par
	\begin{small} Tabela 1. Brak potokowości: wykonanie dwóch rozkazów \end{small} \par
	\end{center}\par

	\begin{center}
	\begin{tabular}{|c|c|c|c|c|c|c|c|} \hline
	Numer 		 & \multicolumn{7}{c|}{Stan potoku} \\
	rozkazu & \multicolumn{7}{c|}{} \\ \hline
	1 & IF & ID & EX & \cellcolor{yellow} MEM & WB & & \\ \hline
	2 & & IF & ID & \cellcolor{yellow} EX & MEM & WB & \\ \hline
	3 & & & IF & \cellcolor{yellow} ID & EX & MEM & WB \\ \hline
	4 & & & & \cellcolor{yellow} IF & ID & EX & MEM \\ \hline
	5 & & & & \cellcolor{yellow} & IF & ID & EX \\ \hline \hline
	Cykl zegara & 1 & 2 & 3 & \cellcolor{yellow} 4 & 5 & 6 & 7 \\ \hline
	\end{tabular}\par
	\begin{small} Tabela 2. Potokowe wykonanie 5 rozkazów \end{small}
	\end{center}
\par
%
\indent
	Poza potokowością, do ILP używa się również innych technik:
	\begin{itemize}
	\item \emph{Superskalarność}: podobnie jak w~potokowości, każdemu krokowi odpowiada co najmniej jeden potok.

	\begin{center}
	\begin{tabular}{|c|c|c|c|c|c|c|c|} \hline
	Numer 		 & \multicolumn{7}{c|}{Stan potoku} \\
	rozkazu & \multicolumn{7}{c|}{} \\ \hline
	1 & IF & ID & EX & \cellcolor{yellow} MEM & WB & & \\ \hline
	2 & IF & ID & EX & \cellcolor{yellow} MEM & WB & & \\ \hline
	3 & & IF & ID & \cellcolor{yellow} EX & MEM & WB & \\ \hline
	4 & & IF & ID & \cellcolor{yellow} EX & MEM & WB & \\ \hline
	5 & & & IF & \cellcolor{yellow} ID & EX & MEM & WB \\ \hline
	6 & & & IF & \cellcolor{yellow} ID & EX & MEM & WB \\ \hline
	7 & & & & \cellcolor{yellow} IF & ID & EX & MEM \\ \hline
	8 & & & & \cellcolor{yellow} IF & ID & EX & MEM \\ \hline
	9 & & & & \cellcolor{yellow} & IF & ID & EX \\ \hline \hline
	10 & & & & \cellcolor{yellow} & IF & ID & EX \\ \hline \hline
	Cykl zegara & 1 & 2 & 3 & \cellcolor{yellow} 4 & 5 & 6 & 7 \\ \hline
	\end{tabular}\par
	\begin{small} Tabela 3. Potokowe wykonanie 10 rozkazów na jednostce superskalarnej\end{small}
	\end{center}
	Jak łatwo zauważyć, superskalarny CPU jest w~stanie przetworzyć nawet wektor instrukcji. Warunkiem koniecznym do osiągnięcia takiego
	wyniku jest brak zależności pomiędzy wynikiem wcześniejszej a~następnej instrukcji.

	\item \emph{Wykonanie poza kolejnością} (ang. \emph{Out of Order execution, OoO}): technika ortogonalna do poprzednich. Polega na 
		wykonaniu tych rozkazów procesora, które mogą użyć bieżącej wartości z~któregoś z~jego rejestrów.
	\item \emph{Przemianowanie rejestrów} (ang. \emph{Register Renaming}): zapobieganie sekwencyjnemu wykonaniu rozkkazów poprzez użycie 
		aktualnie dostępnych rejestrów.
	\item \emph{Wykonanie spekulatywne} (ang. \emph{Speculative execution}): wykonanie instrukcji (lub ich części), które znajdują się
		za skokiem warunkowym przed wynikiem sprawdzenia tego warunku.
	\item \emph{Przewidywanie skoku} (ang. \emph{Branch prediction}): mechanizm wspomagający wykonanie spekulatywne. Polega na wykonaniu
	 ciągu rozkazów, który \emph{może} być wykonany po skoku warunkowym.
	\end{itemize}
\par
%
\indent
	Wyżej wymienione techniki zrównoleglenia wykonania rozkazów napotykają jednak na szereg problemów, przede wszystkim na problem
	zależności od danych.
	Prosty przykład, który ilustruje ten problem:
	\begin{flushright}
	\begin{tabular}{l l}\\
		$x \gets y + z$ & \hspace{4.2cm} (1)\\
		$t \gets u + v$ & \hspace{4.2cm} (2)\\
		$l \gets x \times t$ & \hspace{4.2cm} (3)\\
	\end{tabular}\\
	\end{flushright}
	Jak łatwo zauważyć, wartość komórki $l$~zależy od wyliczonych wartości komórek $x$~i~$t$, a~więc aby otrzymać prawidłowy wynik
	instrukcji (3), jej wykonanie musi poprzedzić wyliczony wcześniej wynik instrukcji (1) i~(2).
\par
%
\indent
	Innym poważnym problemem, z~jakim można się obecnie spotkać, jest dostęp do pamięci RAM: od początku lat 90. XX~wieku różnica
	pomiędzy częstotliwością taktowania procesora, a~pamięci RAM zwiększyła się drastycznie. Już wtedy zaczęto zauważać
	ten problem, przyrównując go, z~punktu widzenia procesora, do uderzenia w~mur (ang. \emph{Memory Wall}). Aby ominąć ten problem,
	w~procesorach umieszczano pamięć podręczną (ang. \emph{cache}), która była taktowana tak jak sam procesor (cache pierwszego poziomu, L1 cache),
	lub częstotliwością niewiele odbiegającą od częstotliwości taktowania procesora (L2 cache). Oczywiście ilość pamięci podręcznej
	była nieporównywalnie mniejsza od ilości pamięci RAM, tak więc pojawienie się pamięci cache wymusiło stosowanie technik, które nie
	były potrzebne dotychczas.
\par
%
\indent
	Okazuje się więc, że zwiększanie częstotliwości taktowania procesora nie wpływa na wzrost jego wydajności w~takim stopniu, jak ILP.
	Należy również pamiętać o~zależności temperatury wydzielanej przez jednostkę centralną w~stosunku do wzrostu częstotliwości taktowania,
	która rośnie wykładniczo.
	%Co więcej, sugerowany przez producentów wzrost wydajności procesora jest traktowany jako mit (ang. \emph{megahertz myth}).
\par
%
\indent
	Stosowanie coraz bardziej zaawansowanych technik (jak również uniknięcie przegrzania jednostki centralnej)
	wymusiło miniaturyzację samego procesora. Następnym krokiem było umieszczenie w~jednostce
	centralnej (CPU) kilku procesorów, zwanych \emph{rdzeniami} (ang. \emph{cores}). Wraz z~pojawieniem się wielordzeniowej jednostki centralnej,
	w~celu efektywnego jej wykorzystania, należy uwzględnić wszystkie problemy związane z~programowaniem równoległym, w~szczególności
	wykonanie aplikacji, w~celu efektywnego wykorzystania zasobów, powinno być zrównoleglone w~jak największym stopniu.
\par

\subsection{Ewolucja systemów operacyjnych}
%
\indent
	Michael J.~Flynn stworzył klasyfikację systemów komputerowych oraz programów, zwaną \emph{taksonomią Flynna}, w~której ze względu na liczbę
	strumieni rozkazów oraz danych, wyróżniamy cztery klasy:
	\begin{center}
	\begin{tabular}{|c|c|c|} \hline
													 & Jeden strumień instrukcji	 & Wiele strumieni instukcji \\
													 & (\emph{Single Instruction}) & (\emph{Multiple Instruction}) \\\hline
			Jeden                &     												 &      \\
			strumień danych      &  SISD										   & MISD 			\\
			(\emph{Single Data}) &                             & \\\hline
			Wiele                &     												 &      \\
			strumieni danych     & SIMD                        & MIMD \\
			(\emph{Multiple Data}) & 										 & 			\\\hline
	\end{tabular}
	\begin{small} Tabela 4. Taksonomia Flynna \end{small}
	\end{center}
	Klasy te możemy scharakteryzowac w~następujący sposób:
	\begin{itemize}
		\item SISD jest klasą programów wykonywanych sekwencyjnie na maszynie sekwencyjnej (skalarnej);
		\item SIMD odpowiada programom wykonywanym na maszynie wektorowej (superskalarnej) lub na procesorach przetwarzających sygnały (DSP);
		\item MISD jest klasą maszyni i~programów, która ma zastosowanie tam, gdzie liczy się minimalizacja błędów;
		\item MIMD odpowiada programom wykonywanym na wielu procesorach (wielordzeniowe, klastry, itp.).
	\end{itemize}
\par
%
\indent
	Aplikację można roapztrywać jako zbiór zadań, które wykonywane są w~określonym kontekście. Może zdarzyć się również, że pewne zadania
	mogą być wykonane po zakończeniu działania innych. Może się zdarzyć, że pewne zadania korzystają ze wspólnych zasobów.
	Systemy operacyjne udostępniają podstawową abstrakcję dla zadań: \emph{procesy}. Wykonanie procesu odbywa się w~jednym lub wielu \emph{wątkach}
	Systemy operacyjne można podzielić na analogiczne klasy do tych z~taksonomii Flynna, ale ze względu na sposób zarządzania zadaniami.
\par
%
\subsection{Motywacja} 
%
\indent
	O~ile ILP jest z~reguły przezroczyste dla programisty (optymalizacja jest dokonywana przez kompilator lub jest wykonywana w~trakcie wykonywania programu
	przez jednostkę centralną),	o~tyle programista musi zaprojektować (i~zaimplementować) aplikację tak, aby mogła korzystać z~wielordzeniowej
	architektury jednostki centralnej.
\par
%
\indent
	Współczesne systemy operacyjne udostępniają programiście narzędzia, które mogą być użyte do maksymalnego wykorzystania 
	wielordzeniowej architektury CPU: są to \emph{procesy} oraz \emph{wątki}.
\par
%
\indent
	Aplikacja jest najczęściej zbiorem różnych zadań wykorzystujących wspólne zasoby, tak więc z~punktu widzenia systemu operacyjnego 
	powinna działać w~obrębie procesu, który działa na udostępnianych przez system zasobach (pamięć, deskryptory plików, czas wykonania, itp.).
	Poszczególne zadania realizowane w~ramach aplikacji winny być wątkami korzystającymi ze wspólnych zasobów.
	Ze względu na naturę wątków, niektóre zadania wymagają wątków przestrzeni jądra (zapewniające skalowalność aplikacji), zaś inne wymagają
	wątków przestrzeni użytkownika (wątek jest przyporządkowany do jednego rdzenia) lub \emph{włókien} (ang. \emph{fibers}, autonomiczne fragmenty
	aplikacji, które dobrowolnie przekazują między siebie sterowanie).
\par
%
\indent
	Niektóre systemy operacyjne (NetBSD, OpenSolaris) udostępniają wątki mieszane, gdzie użytkownik decyduje, z~jakim rodzajem wątku 
	chce związać konkretne zadanie.
\par
%
\indent
	W systemie Linux użytkownik ma do dyspozycji wątki przestrzeni jądra \linebreak (\emph{PThreads}), zgodne ze standardem POSIX0.1.
	Dostępne są biblioteki implementujące włókna (\emph{GNU~Pth}, \emph{State~Threads}). Programista decyduje o~rodzaju użytych wątków.
\par
%
\indent
	W systemie Windows zaimplementowane są wątki przestrzeni jądra, dodatkowo można skorzystać z~włókien, które dostarcza producent systemu.
	Niestety, ich implementacja nie jest zgodna ze standardem POSIX. Również tutaj decyzja o~wyborze rodzaju wątku dla zadania należy do użytkownika.
\par
%
\subsection{Cel}
%
\indent
	Cel tej pracy to zbadanie możliwości elastycznego zarządania wątkami mieszanymi w~taki sposób, aby zminimalizować liczbę odwołań systemowych.
	Co więcej, wątki mają być zarządzane w~sposób przezroczysty dla użytkownika (programisty). 
\par
%

\newpage
\section{Podstawy}

\subsection{Koprocedury}
%
\indent
	Procedurę można przedstawić jako ciąg instrukcji $P=\langle c_1,c_2,\ldots ,c_n\rangle$, zaś jej wywołanie jako 
	$C=\langle c_1^P,\ldots,c_m^P\rangle, \mathrm{ } (m \leq n)$~--~jej
	podciąg. Dla dwóch wywołań 
	\begin{displaymath}
	\begin{array}{l}
	C_1= \langle c_1^1,c_2^1,\ldots ,c_k^1 \rangle \\
	\mathrm{oraz}\\
	C_2= \langle c_1^2,c_2^2,\ldots ,c_l^2 \rangle \mathrm{~} (k,l > 0), 
	\end{array}
	\end{displaymath}
	mamy $c_1^1 = c_2^1$, zaś ostatnie elementy tych ciągów wcale nie muszą być równe.
\par
\indent
	Każda procedura posiada specyficzne dla niej obiekty zawarte w~jej \emph{rekordzie aktywacji},
	umieszczonym na stosie tych rekordów. Rekord aktywacji zawiera stan procedury, który znika wraz 
	z~jej zakończeniem. Tak więc, stan procedury jest ulotny.
\par
	Koprocedury są ,,uogólnionymi'' procedurami. O~ile procedury posiadają własne zmienne
	przechowywane na stosie oraz mogą dostarczyć dokładnie jedną wartość po zakończeniu
	działania, o~tyle koprocedury:
	\begin{itemize}
	\item posiadają własny stos wywołań;
	\item mogą przekazać do strony wywołującej co najmniej jedną wartość.
	\end{itemize}
	Powyższe właściwości koprocedur predestynują je jako naturalna podstawę dla wątków:
	nie są one tak ulotne jak procedury, gdyż po zwróceniu wyniku ich stan jest zapamiętywany,
	a~kolejne wywołanie koprocedury jest jej kontynuacją od miejsca poprzedzającego przekazanie
	ostatniego wyniku. Co więcej, koprocedura może wywoływać prcedury (jak również inne koprocedury)
	we własnym kontekście, a~więc, przy minimalnym nakładzie ze strony programisty, możliwe jest
	zaimplementowanie TLS dla bieżącej koprocedury.
\par

\indent
	W wielowątkowym systemie operacyjnym, proces jest jednostką o~chronionym dostępie do 
	procesorów, innych procesów, plików oraz innych zasobów wejścia-wyjścia. Ponadto, system
	traktuje go jako autonomiczny byt, który posiada własną przestrzeń adresową (wirtualna 
	przestrzeń adresowa), oraz czas wykonania.
\par
%
\indent
	W~takim systemie wątek jest przyporządkowany do dokładnie jednego procesu, a~sam proces 
	może zawierać co najmniej jeden wątek. Wątki w~obrębie procesu dzielą tę samą przestrzeń
	adresową, jak i~pozostałe zasoby. Czas wykonania wątków jednego procesu jest równy czasowi
	jego wykonania. Każdemu z~nich są przyporządkowane:
	\begin{itemize}
	\item stan (uruchomiony, gotowy, itp.);
	\item kontekst (gdy nie jest aktualnie wykonywany);
	\item stos wywołań;
	\item prywatna przestrzeń adresowa (ang. \emph{Thread Local Storage, TLS}).
	\end{itemize}
\par
\subsection{Wątki przestrzeni jądra}

%
\indent
	Wątki przestrzeni jądra (ang. \emph{Kernel--Level Threads, KLT}) są zarządzane w~całości przez system operacyjny. 
	Większość obecnych systemów operacyjnych oferuje KLT użytkownikowi. Czasami wątki przestrzeni jądra są nazywane
	\emph{procesami wagi lekkiej} (ang. \emph{Lightweight Proccess}), gdyż przełączanie pomiędzy nimi jest o~wiele
	szybsze niż pomiędzy procesami, zaś każdy wątek otrzymuje od systemu określoną porcję czasu procesora.
	KLT realizują model 1:1. Odzwierciedla on przyporządkowanie \emph{jednego wątku użytkownika dla jednego wątku jądra}.
\par
%
\indent
	Podstawowa korzyść, jaką można uzyskać z~tego powodu, to nieblokujący dostęp do urządzeń wejścia-wyjścia:
	jeśli jeden z~wątków oczekuje na dane z~IO, wykonanie pozostałych wątków nie jest wstrzymywane.
	Z~tego powodu aplikacje, które w~trakcie działania często odwołują się do urządzeń zewnętrznych, korzystają
	z~KLT.
\par
%
\indent
	Inną korzyścią wątków przestrzeni jądra jest rozdystrybuowanie ich na wiele rdzeni. Mechanizm ten jest przezroczysty
	dla użytkownika. Dzięki niemu, niezależne obliczenia mogą wykonywać się równocześnie, co może znacząco wpłynąć na
	wydajność aplikacji.
\par
%
\indent
	Należy pamiętać, że KLT są zarządzane w~całości przez system operacyjny. Wobec tego, zmiana stanu wątku, jego uruchomienie,
	wstrzymanie, jak również sposób zarządzania pulą wątków jest w~pełni zależna od jądra systemu i~użytkownik jest skazany
	na decyzje systemu operacyjnego odnośnie podstawowych operacji na wątkach. Co więcej, taki sposób zarządzania wątkami,
	jest obarczony kosztami związanymi z~komunikacją z~jądrem systemu.
\par
%
\indent
	Użytkownik korzystający z~wątków przestrzeni jądra może spotkać się z~problemem \emph{wyścigów} (ang. \emph{Race Conditions}):
	dwa wątki w~tym samym czasie wykonują operacje na tych samych zasobach. Ponieważ system operacyjny przerywa pracę wątków w~sposób
	niezależny od działania poszczególnych wątków, może zdarzyć się sytuacja, gdy operacje jednego wątku \emph{przeplatają się} z~operacjami
	drugiego wątku (równie dobrze może to być o~wiele większa liczba wątków!) na tym samym zasobie. W~takiej sytuacji, wynik działania wątków
	może prowadzić do nieprzewidywalnych wyników, które mogą wypaczyć wynik działania aplikacji, lub nawet doprowadzić do przerwania jej wykonania.
\par
%
\indent
	Aby zaradzić temu problemowi, stworzony został mechanizm \emph{wzajemnego wykluczania} (ang. \emph{Mutual Exclusion, mutex}):
	jeśli dany zasób jest wolny, wówczas dostęp do niego otrzymuje dokładnie jeden wątek, zaś inne wątki czekają na zwolnienie tego zasobu.
	Wątki mogą być uśpione lub oczekiwać aktywnie na przydzielenie zasobu, jednakże są one w~jakiś sposób zablokowane.  
	Okazuje się, że mechanizm wzajemnego wykluczania wiąże się z~różnego rodzaju zagrożeniami. Najpoważniejsze z~nich to zakleszczenie 
	(ang. \emph{deadlock}) i~zagłodzenie (ang. \emph{starvation}).
\par
%
\indent
	Zakleszczenie występuje w~momencie gdy jeden wątek wykonuje operacje na zasobie A, a~następnie (nie zwalniając A) zgłasza zapotrzebowanie
	na zasób B, który jest zarezerwowany dla drugiego wątku, który oczekuje na zwolnienie zasobu A.
\par
%
\indent
	Zagłodzenie wątku występuje wtedy, gdy pewien wątek oczekuje na dostęp do zasobu, lecz nie jest on do tego zasobu dopuszczony.
\par

\subsection{Wątki przestrzeni użytkownika}
%
\indent
	Wątki przestrzeni użytkownika (ang. \emph{User--Level Threads, ULT}) nie są widoczne dla systemu operacyjnego w~taki sposób,
	jak ma to miejsce w~wypadku KLT. Podstawowa różnica pomiędzy tymi dwoma rodzajami wątków polega na podziale czasu pomiędzy
	poszczególne wątki: uruchomienie (wznawianie) oraz przerywanie działania KLT jest dokonywane przez system operacyjny, zaś 
	zarządzanie działaniem ULT jest zadaniem dla użytkownika. Strategie uruchamiania poszczególnych wątków przestrzeni jądra
	(ang. \emph{scheduling}) jest również odpowiedzialnością systemu operacyjnego, co w~przypadku ULT jest także powinnością 
	użytkownika. Wątki przestrzeni użytkownika są z~tego powodu o~wiele lżejsze: przełączanie pomiędzy nimi nie jest tak czasochłonne
	jak w~wypadku KLT. Co więcej, zarządzanie ULT może być bardziej dopasowane do charakteru aplikacji: niektóre wątki potrzebują
	prostego zarządzania (pierwszy wątek, drugi, \ldots, $n$-ty, pierwszy, \ldots -- algorytm \emph{Round-Robin}),
	inne aplikacje wymagają nadawania poszczególnym wątkom priorytetów, itp. Co więcej, przezroczystość ULT dla systemu nie zakłóca
	jego własnej polityki zarządzania wątkami. Model, który odpowiada wątkom przestrzeni użytkownika, to N:1.
	ULT są nazywane \emph{zielonymi wątkami} (ang. \emph{Green Threads}).
\par
%
\indent
	Zaletą tego rozwiązania jest także łatwość uniknięcia zjawiska wyścigów. Ponieważ wszystkie ULT są wykonywane
	w~ramach jednego procesu, wątek posiada wyłączny dostęp do wszystkich jego zasobów,
	nie modyfikowany przez żaden inny wątek.
\par
%
\indent
	Jeśli jeden z~wątków przestrzeni
	użytkownika zostanie zablokowany (np. podczas wykonywania operacji na IO), wówczas działanie innych wątków będzie również zablokowane.
	Aby uniknąć takich sytuacji, operacje blokujące są często opakowane w~taki sposób, by nie blokować wykonania wątku (\emph{jacketing}).
\par
%
\indent
	Ważną cechą ULT jest ich niezależność od systemu operacyjnego, w~jakim pracują: nie potrzebują one mechanizmu wywłaszczania
	(ang. \emph{preemption}), gdyż kontrola nad wykonaniem wątku jest wykonywana po stronie użytkownika.
\par
%
\subsection{Wątki mieszane}
%
\indent
	Wątki mieszane są implementacją modelu N:M. Wątki w~tym modelu są zarządzane zarówno 
	przez system operacyjny, jak też użytkownika. Ich celem jest połączenie zalet KLT i ULT. Istnieją implementacje, które
	modyfikują system operacyjny, jak również takie, które nie ingerują w~jego strukturę. Opisana w~tej pracy implementacja
	wątków mieszanych stosuje drugą metodę.
\par
\subsection{Wzajemne wykluczanie}
%
\indent
	Jak wspomniano wcześniej, wzajemne wykluczanie jest mechanizmem używanym w~celu uniknięcia zjawiska wyścigów.
	Aby wykluczyć taką sytuację, wątki wykonują działania na współdzielonym zasobie w~ramach tzw. \emph{sekcji krytycznej}
	(ang. \emph{critical section}).
\par
%
\indent
	Sekcję krytyczną można zapisać w~następujący sposób:
	\begin{verbatim}

    Wejście_do_sekcji_krytycznej();

    Sekcja_krytyczna(); // wykonanie operacji
                        // na współdzielonych zasobach

    Wyjście_z_sekcji_krytycznej();
	\end{verbatim}
	
	Operacje wejścia i~wyjścia z~sekcji krytycznej muszą spełniać następujące warunki:
	\begin{itemize}
	\item[{\bf M}] Wzajemne wykluczanie : dokładnie jeden wątek znajduje się w~sekcji krytycznej
	(Jeśli nastąpi przełączenie tego wątku na inny -- jest on nadal w~sekcji krytycznej, a~więc inne wątki
	w~dalszym ciągu muszą czekać na jego wyjście z~sekcji krytycznej).
	\item[{\bf P}] Postęp (ang. \emph{Progress}): jeśli żaden wątek nie znajduje się w~sekcji krytycznej, a~istnieją wątki oczekujące na wejście
	do własnych sekcji krytycznych, to decyzja o~wejściu do sekcji krytycznej zapada pomiędzy wątkami, które oczekują na wejście
	lub wychodzą z~sekcji krytycznej. Co więcej, decyzja ta musi zapaść (nie może być odraczana w~nieskończoność).
	\item[{\bf B}] Ograniczone czekanie (ang. \emph{Bounded waiting}): wątek znajdzie się w~sekcji krytycznej najpóźniej po określonej
	liczbie prób.
	\end{itemize}
\par
\subsubsection{Algorytm Petersona}
%
\indent
	Algorytm jest rozwiązaniem problemu sekcji krytycznej dla dwóch wątków. 
\par


%\end{document}
%
%\newpage
%\section{Strategie (polityki) zarządzania wątkami}

\subsection{Zarządzanie wątkami przestrzeni jądra}

\subsubsection{Strategie zaimplementowane w Linuksie.}

\paragraph{CFS (\emph{Complete Fair Strategy})}

\subsubsection{Strategie zaimplementowane w systemie FreeBSD}

\subsubsection{MacOS(?)}

\subsection{Zarządca wątków przestrzeni użytkownika}

\subsection{Zarządca włókien}

\subsubsection{Ad Hoc}

\subsubsection{Kolejka priorytetowa (na wzór CFS)}

\subsubsection{Heurystyki?}

%
%\newpage
%\section{Mechanizmy użyte do implementacji strategii}

\subsection{Drzewa zbalansowane}

\subsection{Drzewa samoorganizujące się}

\subsection{Kopce}


%
%\newpage
%\section{Implementacja}

\subsection{Koprocedury -- biblioteka {\bf {\tt libcoro}}}

\subsection{Magazyn koprocedur i kontekstów -- {\bf {\tt context\_manager}}}

\subsection{Zarządca wątków -- {\bf {\tt thread\_manager}}}


%
%\newpage
%\section{Testowanie}

\subsection{Rodzaje testów}

\subsubsection{Testy jednostkowe}

\subsubsection{Testy wydajnościowe}

\subsection{Użyte narzędzia (uzasadnienie wyboru konkretnych narzędzi)}

\subsubsection{Śledzenie wykonania wątków}

\subsubsection{Valgrind jako przykład narzędzia do kontroli pamięci
							 (wycieki pamięci, pamięć podręczna procesora, 
							 \emph{race conditions})}

\subsubsection{CppUnit jako przykład środowiska użytego do testów jednostkowych}

\subsection{Opis aplikacji użytych do testów wydajnościowych}


%
%\newpage
%\section{Podsumowanie}

%
%\newpage
%\begin{thebibliography}{mmi}

\bibitem[Knu1]{kn} Donald E. Knuth: \emph{The Art of Computer Programming}, US\&A 199?
\bibitem[Ble1]{ble} Donald E. Knuth: \emph{The Art of Computer Programming}, US\&A 199?
\end{thebibliography}

%
\end{document}
