\section{Wstęp}
%
\indent 
	Od połowy lat 80. XX~wieku zwiększano moc obliczeniową procesorów za pomocą \emph{równoległego wykonania rozkazów procesora}
	(ang. \emph{instruction-level parallelism, ILP}). Wykonanie rozkazu procesora można podzielić na pewną liczbę kroków
	(od kilku do nawet kilkudziesięciu!),	a~następnie przetwarzać je niezależnie. 
	Stosując takie podejście, rozkazy mogą się \emph{skalować}, np. jeśli rozkazy A~i~B~wykonują się osobno w~ciągu 10 cykli każdy, to,
	przy spełnieniu odpowiednich wymogów, wykonanie rozkazu A, a~następnie rozkazu B~może zająć procesorowi jedynie 11 cykli.
\par
%
\indent
	Techniki, za pomocą których uzyskuje się ILP:
	\begin{itemize}
	\item \emph{potokowość}: każdy potok odpowiada dokładnie jednemu krokowi przetwarzania rozkazu procesora.
	\item \emph{Superskalarność}: podobnie jak w~potokowości, każdemu krokowi odpowiada co najmniej jeden potok.
	\item \emph{Wykonanie poza kolejnością} (ang. \emph{Out of Order execution, OoO}): technika ortogonalna do poprzednich. Polega na 
		wykonaniu tych rozkazów procesora, które mogą użyć bieżącej wartości z~któregoś z~jego rejestrów.
	\item \emph{Przemianowanie rejestrów} (ang. \emph{Register Renaming}): zapobieganie sekwencyjnemu wykonaniu rozkkazów poprzez użycie 
		aktualnie dostępnych rejestrów.
	\item \emph{Wykonanie spekulatywne} (ang. \emph{Speculative execution}): wykonanie instrukcji (lub ich części), które znajdują się
		za skokiem warunkowym przed wynikiem sprawdzenia tego warunku.
	\item \emph{Przewidywanie skoku} (ang. \emph{Branch prediction}): mechanizm wspomagający wykonanie spekulatywne.
	\end{itemize}
\par
%
\indent
	Wyżej opisane techniki zrównoleglenia wykonania rozkazów napotykają jednak na szereg problemów. Oto niektóre z nich:
\par
%
\indent
	Obecnie rozwój procesorów nie polega więc na zwiększaniu częstotliwości ich taktowania -- zamiast tego, jednostka centralna
	składa się z~kilku procesorów, tzw. \emph{rdzeni} (\emph{core}), z~których każdy posiada swoją pamięć podręczną, a~każdy
	rdzeń może być taktowany inną częstotliwością.
	Taki kierunek rozwoju procesorów został również podyktowany rozwojem oprogramowania, które jest na nich najczęściej uruchamiane:
	Prawie każdy współczesny system operacyjny jest systemem wielozadaniowym (współbieżnym), a~więc wielordzeniowe procesory wydają się
	być idealnie dopasowane do wymagań takich systemów.
\par
%
\subsection{Motywacja}
%
\indent
	O~ile system operacyjny jest w~stanie rozpoznać wielordzeniową jednostkę centralną i~efektywnie wykorzystać jej możliwości,
	o~tyle aplikacje korzystające z~wielordzeniowej architektury napoykają na szereg problemów:
	\begin{itemize}
		\item osobne procesy mogą komunikować się między sobą tylko za pośrednictwem systemu operacyjnego: spadek wydajności aplikacji;
		\item zadania w~obrębie jednego procesu mogą być wykonywane szeregowo: brak skalowalności, aplikacja nie wykorzystuje większej liczby
		rdzeni w~systemie;
		\item osobne wątki -- przestrzeni jądra, które korzystają z~wspólnych zasobów: problem wzajemnego wykluczania, który często prowadzi
		do spadku wydajności aplikacji;
		\item wątki przestrzeni użytkownika mogą zniwelować problemy związane z~wzajemnym wykluczaniem, jenakże w~tym wypadku znów wystąpi
		problem braku skalowalności.
	\end{itemize}
\par
%
\indent
	Aplikacja jest najczęściej zbiorem różnych zadań, tak więc z~punktu widzenia systemu operacyjnego powinna działać w~obrębie procesu: 
	bytu posiadającego do własnej dyspozycji zasoby udostępniane przez system (czyli pamięć, deskryptory plików, czas wykonania, itp.).
	Poszczególne zadania realizowane w~ramach aplikacji, ze względu na wielordzeniową architekturę obecnych jednostek centralnych, winny być wątkami.
	Ze względu na naturę wątków, niektóre zadania wymagają wątków przestrzeni jądra (zapewniające skalowalność aplikacji), zaś inne wymagają
	wątków przestrzeni użytkownika (wzajemne wykluczanie można zrealizować o~wiele łatwiej w~przestrzeni użytkownika).
\par
%
\indent
	Niektóre systemy operacyjne (NetBSD, OpenSolaris) udostępniają wątki mieszane, gdzie użytkownik decyduje, z~jakim rodzajem wątku 
	chce związać konkretne zadanie.
\par
%
\indent
	W systemie Linux użytkownik ma do dyspozycji wątki przestrzeni jądra \linebreak (\emph{PThreads}), zgodne ze standardem POSIX0.1, zaś wątki przestrzeni
	użytkownika są dostepne jako osobne biblioteki (np. \emph{GNU Pth}).
\par
%
\indent
	W systemie Windows zaimplementowane są wątki przestrzeni jądra, dodatkowo można skorzystać z~wątków przestrzeni użytkownika,
	zwanych \emph{włóknami} (\emph{fibers}). Również tutaj decyzja o~wyborze rodzaju wątku dla zadania należy do użytkownika.
\par
%
\subsection{Cel}
%
\indent
	Celem głównym tej pracy było zbadanie możliwości połączenia wątków przestrzeni użytkownika z~wątkami przestrzeni jądra w~taki sposób,
	aby zdjąć z~użytkownika ciężar decyzji co do rodzaju wątku dla zadania, a~ponadto zbadanie, czy bez dużej straty efektywności można  
	zmieniać właściwości wątków tak, aby bardziej odpowiadały podzadaniom realizowanym w~zadaniach.
\par
