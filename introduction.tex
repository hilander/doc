\section{Wstęp}
%
\indent 
	Od połowy lat 80. XX~wieku zwiększano moc obliczeniową procesorów za pomocą \emph{równoległego wykonania rozkazów procesora}
	(ang. \emph{instruction-level parallelism, ILP}). Na wykonanie pojedynczego rozkazu składa się wykonanie w~ustalonej kolejności
	określonej liczby kroków. Klasyczna sekwencja składa się z~pięciu kroków:
	\begin{enumerate}
	\item pobranie (ang. \emph{Instruction Fetch, IF});
	\item dekodowanie (ang. \emph{Instruction Decode, ID});
	\item wykonanie (ang. \emph{Execute, EX});
	\item dostęp do pamięci (ang. \emph{Memory Access, MEM});
	\item zapis (ang. \emph{Write Back, WB}).
	\end{enumerate}
	Każdy rozkaz był wykonywany oddzielnie, tj. jednostka centralna wykonywała kroki od IF do WB dla każdego rozkazu osobno, zaś
	podejście zastosowane w ILP polegało na maksymalnym wykorzystaniu części procesora obsługujących poszczególne kroki.
	Jet to \emph{potokowe przetwarzanie rozkazów} (ang. \emph{Pipelining}).
	%Wykonanie rozkazu procesora można podzielić na pewną liczbę kroków
	%(od kilku do nawet kilkudziesięciu!),	a~następnie przetwarzać je niezależnie. 
	%Stosując takie podejście, rozkazy mogą się \emph{skalować}, np. jeśli rozkazy A~i~B~wykonują się osobno w~ciągu 10 cykli każdy, to,
	%przy spełnieniu odpowiednich wymogów, wykonanie rozkazu A, a~następnie rozkazu B~może zająć procesorowi jedynie 11 cykli.
	\begin{center}
	\begin{tabular}{|c|c|c|c|c|c|c|c|c|c|c|} \hline
	& \multicolumn{5}{c|}{Rozkaz 1} & \multicolumn{5}{c|}{Rozkaz 2}  \\ \cline{2-11}
	& IF & ID & \cellcolor{yellow} EX & MEM & WB & IF & ID & EX & MEM & WB\\ \cline{2-11} \cline{2-11}
	Cykl zegara & 1 & 2 & \cellcolor{yellow} 3 & 4 & 5 & 6 & 7 & 8 & 9 & 10 \\ \hline
	\end{tabular}\par
	\begin{small} Tabela 1. Brak potokowości: wykonanie dwóch rozkazów \end{small} \par
	\end{center}\par

	\begin{center}
	\begin{tabular}{|c|c|c|c|c|c|c|c|} \hline
	Numer 		 & \multicolumn{7}{c|}{Stan potoku} \\
	rozkazu & \multicolumn{7}{c|}{} \\ \hline
	1 & IF & ID & EX & \cellcolor{yellow} MEM & WB & & \\ \hline
	2 & & IF & ID & \cellcolor{yellow} EX & MEM & WB & \\ \hline
	3 & & & IF & \cellcolor{yellow} ID & EX & MEM & WB \\ \hline
	4 & & & & \cellcolor{yellow} IF & ID & EX & MEM \\ \hline
	5 & & & & \cellcolor{yellow} & IF & ID & EX \\ \hline \hline
	Cykl zegara & 1 & 2 & 3 & \cellcolor{yellow} 4 & 5 & 6 & 7 \\ \hline
	\end{tabular}\par
	\begin{small} Tabela 2. Potokowe wykonanie 5 rozkazów \end{small}
	%\includegraphics[scale=.7]{pipeline.png}\\
	%\small{Rysunek 1. Potokowe wykonanie rozkazów}
	\end{center}
\par
%
\indent
	Poza potokowością, do ILP używa się również innych technik:
	\begin{itemize}
	\item \emph{Superskalarność}: podobnie jak w~potokowości, każdemu krokowi odpowiada co najmniej jeden potok.
	\item \emph{Wykonanie poza kolejnością} (ang. \emph{Out of Order execution, OoO}): technika ortogonalna do poprzednich. Polega na 
		wykonaniu tych rozkazów procesora, które mogą użyć bieżącej wartości z~któregoś z~jego rejestrów.
	\item \emph{Przemianowanie rejestrów} (ang. \emph{Register Renaming}): zapobieganie sekwencyjnemu wykonaniu rozkkazów poprzez użycie 
		aktualnie dostępnych rejestrów.
	\item \emph{Wykonanie spekulatywne} (ang. \emph{Speculative execution}): wykonanie instrukcji (lub ich części), które znajdują się
		za skokiem warunkowym przed wynikiem sprawdzenia tego warunku.
	\item \emph{Przewidywanie skoku} (ang. \emph{Branch prediction}): mechanizm wspomagający wykonanie spekulatywne. Polega na wykonaniu
	 ciągu rozkazów, który \emph{może} być wykonany po skoku warunkowym.
	\end{itemize}
\par
%
\indent
	Wyżej wymienione techniki zrównoleglenia wykonania rozkazów napotykają jednak na szereg problemów, przede wszystkim na problem
	zależności od danych.
	Prosty przykład, który ilustruje ten problem:
	\begin{flushright}
	\begin{tabular}{l l}\\
		$x \gets y + z$ & \hspace{4.2cm} (1)\\
		$t \gets u + v$ & \hspace{4.2cm} (2)\\
		$l \gets x \times t$ & \hspace{4.2cm} (3)\\
	\end{tabular}\\
	\end{flushright}
	Jak łatwo zauważyć, wartość komórki $l$~zależy od wyliczonych wartości komórek $x$~i~$t$, a~więc aby otrzymać prawidłowy wynik
	instrukcji (3), jej wykonanie musi poprzedzić wyliczony wcześniej wynik instrukcji (1) i~(2).
\par
%
\indent
	Innym poważnym problemem, z~jakim można się obecnie spotkać, jest dostęp do pamięci RAM: od początku lat 90. XX~wieku różnica
	pomiędzy częstotliwością taktowania procesora, a~pamięci RAM zwiększyła się drastycznie. Już wtedy zaczęto zauważać
	ten problem, przyrównując go, z~punktu widzenia procesora, do uderzenia w~mur (ang. \emph{Memory Wall}). Aby ominąć ten problem,
	w~procesorach umieszczano pamięć podręczną (ang. \emph{cache}), która była taktowana tak jak sam procesor (cache pierwszego poziomu, L1 cache),
	lub częstotliwością niewiele odbiegającą od częstotliwości taktowania procesora (L2 cache). Oczywiście ilość pamięci podręcznej
	była nieporównywalnie mniejsza od ilości pamięci RAM, tak więc pojawienie się pamięci cache wymusiło stosowanie technik, które nie
	były potrzebne dotychczas.
\par
%
\indent
	Okazuje się więc, że zwiększanie częstotliwości taktowania procesora nie wpływa na wzrost jego wydajności w~takim stopniu, jak ILP.
	Co więcej, sugerowany przez producentów wzrost wydajności procesora jest traktowany jako mit (ang. \emph{megahertz myth}).
\par
%
\indent
	Stosowanie coraz bardziej zaawansowanych technik wymusiło miniaturyzację samego procesora. Następnym krokiem było umieszczenie w~jednostce
	centralnej (CPU) kilku procesorów, zwanych \emph{rdzeniami} (ang. \emph{cores}). Wraz z~pojawieniem się wielordzeniowej jednostki centralnej,
	w~celu efektywnego jej wykorzystania, należy uwzględnić wszystkie problemy związane z~programowaniem równoległym, w~szczególności
	wykonanie aplikacji, w~celu efektywnego wykorzystania zasobów, powinno być zrównoleglone w~jak największym stopniu.
\par
%
\subsection{Motywacja} 
%
\indent
	O~ile ILP jest z~reguły przezroczyste dla programisty (optymalizacja jest dokonywana przez kompilator lub jest wykonywana \emph{w~locie}
	przez jednostkę centralną),	o~tyle programista musi zaprojektować (i~zaimplementować) aplikację tak, aby mogła korzystać z~wielordzeniowej
	architektury jednostki centralnej.
\par
%
\indent
	Współczesne systemy operacyjne udostępniają programiście narzędzia, które mogą być użyte do maksymalnego wykorzystania 
	wielordzeniowej architektury CPU: są to \emph{procesy} oraz \emph{wątki}.
\par
%
\indent
	Aplikacja jest najczęściej zbiorem różnych zadań wykorzystujących wspólne zasoby, tak więc z~punktu widzenia systemu operacyjnego 
	powinna działać w~obrębie procesu, który działa na udostępnianych przez system zasobach (pamięć, deskryptory plików, czas wykonania, itp.).
	Poszczególne zadania realizowane w~ramach aplikacji winny być wątkami korzystającymi ze wspólnych zasobów.
	Ze względu na naturę wątków, niektóre zadania wymagają wątków przestrzeni jądra (zapewniające skalowalność aplikacji), zaś inne wymagają
	wątków przestrzeni użytkownika (wątek jest przyporządkowany do jednego rdzenia) lub \emph{włókien} (ang. \emph{fibers}, autonomiczne fragmenty
	aplikacji, które dobrowolnie przekazują między siebie sterowanie).
\par
%
\indent
	Niektóre systemy operacyjne (NetBSD, OpenSolaris) udostępniają wątki mieszane, gdzie użytkownik decyduje, z~jakim rodzajem wątku 
	chce związać konkretne zadanie.
\par
%
\indent
	W systemie Linux użytkownik ma do dyspozycji wątki przestrzeni jądra \linebreak (\emph{PThreads}), zgodne ze standardem POSIX0.1.
	Dostępne są biblioteki implementujące włókna (\emph{GNU~Pth}, \emph{State~Threads}). Programista decyduje o~rodzaju użytych wątków.
\par
%
\indent
	W systemie Windows zaimplementowane są wątki przestrzeni jądra, dodatkowo można skorzystać z~włókien, które dostarcza producent systemu.
	Niestety, ich implementacja nie jest zgodna ze standardem POSIX. Również tutaj decyzja o~wyborze rodzaju wątku dla zadania należy do użytkownika.
\par
%
\subsection{Cel}
%
\indent
	Cel tej pracy to zbadanie możliwości elastycznego zarządania wątkami mieszanymi w~taki sposób, aby zminimalizować liczbę odwołań systemowych.
	Co więcej, wątki mają być zarządzane w~sposób przezroczysty dla użytkownika (programisty). 
\par
