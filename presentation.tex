\documentclass[10pt]{beamer}
\usetheme{Warsaw}
\setbeamertemplate{navigation symbols}{}
\usepackage[OT1]{fontenc}
\usepackage[utf8]{inputenc}
\usepackage{polski}
\author[Mariusz Barycz]{Mariusz Barycz\\Praca magisterska}
\date{Wrocław, 14 czerwca 2012}
\title{Analiza metod zarządzania wątkami mieszanymi}

\begin{document}
\begin{frame}
\titlepage
\end{frame}

\frame{
\frametitle{Plan prezentacji}
\tableofcontents
}
\section{Wstęp}
\subsection{Motywacja}
\frame
{
\begin{block}{Rodzaje wątków}
  \begin{center}
  \begin{tabular}{p{3cm}|p{3cm}|p{3cm}}
    \multicolumn{1}{c|}{\bf jądra} & \multicolumn{1}{c|}{\bf użytkownika} & \multicolumn{1}{c}{\bf mieszane} \\ \hline
    {zarządzane przez system operacyjny} & {zarządzane przez użytkownika} & {zarządzane przez użytkownika} \\ \hline
    obciążają system operacyjny & obciążają program użytkownika & obciążają program użytkownika \\ \hline
    intensywnie rozwijane & nie rozwijane & brak dostępnej implementacji \\
  \end{tabular}
\end{center}

\end{block}
}

\frame{
\begin{block}{Wątki mieszane -- oczekiwane korzyści}
\end{block}
}

\subsection{Cel pracy}
\frame{
\begin{block}{Docelowa biblioteka włókien}
\end{block}
}

\section{System operacyjny}
\subsection{Procesy i~wątki}
\frame{
\begin{block}{Procesy}
\end{block}

\begin{block}{Linux}
\end{block}
}

\frame{
\begin{block}{Wątki przestrzeni jądra}
\end{block}

\begin{block}{Włókna}
\end{block}
}
\subsection{Synchronizacja}
\frame
{
\begin{block}{Problemy związane ze współbieżnością}
\end{block}
\begin{block}{Mechanizmy synchronizacji}
\end{block}
}

\section{Rozwiązanie}
\subsection{Opis biblioteki}
\frame{
\begin{block}{Podstawowe klasy biblioteki}
\end{block}
\begin{block}{Wykorzystanie wątków przestrzeni jądra}
\end{block}
}

\section{Podsumowanie}
\subsection{Testy}
\frame
{
\begin{block}{Wyniki testów}
\end{block}
}

\frame
{

}

\frame{
\center{Dziękuję za uwagę.}
}

%\begin{block}{Title of the block}
%\begin{exampleblock}{Title  of the block}
%\begin{alertblock}{Title  of the block}
\end{document}
