%\documentclass{mwart}
%\usepackage{polski}
%\begin{document}
\section{Podstawy}
\indent
	W wielowątkowym systemie operacyjnym, proces jest jednostką o~chronionym dostępie do 
	procesorów, innych procesów, plików oraz innych zasobów wejścia-wyjścia. Ponadto, system
	traktuje go jako autonomiczny byt, który posiada własną przestrzeń adresową (wirtualna 
	przestrzeń adresowa), oraz czas wykonania.
\par
%
\indent
	W~takim systemie wątek jest przyporządkowany do dokładnie jednego procesu, a~sam proces 
	może zawierać co najmniej jeden wątek. Wątki w~obrębie procesu dzielą tę samą przestrzeń
	adresową, jak i~pozostałe zasoby. Czas wykonania wątków jednego procesu jest równy czasowi
	jego wykonania. Każdemu z~nich są przyporządkowane:
	\begin{itemize}
	\item stan (uruchomiony, gotowy, itp.);
	\item kontekst (gdy nie jest aktualnie wykonywany);
	\item stos wywołań;
	\item prywatna przestrzeń adresowa (ang. \emph{Thread Local Storage, TLS}).
	\end{itemize}
\par
\subsection{Koprocedury}
%
\indent
	Koprocedury są ,,uogólnionymi'' procedurami. O~ile procedury posiadają własne zmienne
	przechowywane na stosie oraz mogą dostarczyć dokładnie jedną wartość po zakończeniu
	działania, o~tyle koprocedury:
	\begin{itemize}
	\item posiadają własny stos wywołań;
	\item mogą przekazać do strony wywołującej co najmniej jedną wartość.
	\end{itemize}
	Powyższe właściwości koprocedur predestynują je jako naturalna podstawę dla wątków:
	nie są one tak ulotne jak procedury, gdyż po zwróceniu wyniku ich stan jest zapamiętywany,
	a~kolejne wywołanie koprocedury jest jej kontynuacją od miejsca poprzedzającego przekazanie
	ostatniego wyniku. Co więcej, koprocedura może wywoływać prcedury (jak również inne koprocedury)
	we własnym kontekście, a~więc, przy minimalnym nakładzie ze strony programisty, możliwe jest
	zaimplementowanie TLS dla bieżącej koprocedury.
\par

\subsection{Wątki przestrzeni jądra}

%
\indent
	Wątki przestrzeni jądra (ang. \emph{Kernel--Level Threads, KLT}) są zarządzane w~całości przez system operacyjny. 
	Większość obecnych systemów operacyjnych oferuje KLT użytkownikowi. Czasami wątki przestrzeni jądra są nazywane
	\emph{procesami wagi lekkiej} (ang. \emph{Lightweight Proccess}), gdyż przełączanie pomiędzy nimi jest o~wiele
	szybsze niż pomiędzy procesami, zaś każdy wątek otrzymuje od systemu określoną porcję czasu procesora.
	KLT realizują model 1:1. Odzwierciedla on przyporządkowanie \emph{jednego wątku użytkownika dla jednego wątku jądra}.
\par
%
\indent
	Podstawowa korzyść, jaką można uzyskać z~tego powodu, to nieblokujący dostęp do urządzeń wejścia-wyjścia:
	jeśli jeden z~wątków oczekuje na dane z~IO, wykonanie pozostałych wątków nie jest wstrzymywane.
	Z~tego powodu aplikacje, które w~trakcie działania często odwołują się do urządzeń zewnętrznych, korzystają
	z~KLT.
\par
%
\indent
	Inną korzyścią wątków przestrzeni jądra jest rozdystrybuowanie ich na wiele rdzeni. Mechanizm ten jest przezroczysty
	dla użytkownika. Dzięki niemu, niezależne obliczenia mogą wykonywać się równocześnie, co może znacząco wpłynąć na
	wydajność aplikacji.
\par
%
\indent
	Należy pamiętać, że KLT są zarządzane w~całości przez system operacyjny. Wobec tego, zmiana stanu wątku, jego uruchomienie,
	wstrzymanie, jak również sposób zarządzania pulą wątków jest w~pełni zależna od jądra systemu i~użytkownik jest skazany
	na decyzje systemu operacyjnego odnośnie podstawowych operacji na wątkach. Co więcej, taki sposób zarządzania wątkami,
	jest obarczony kosztami związanymi z~komunikacją z~jądrem systemu.
\par
%
\indent
	Użytkownik korzystający z~wątków przestrzeni jądra może spotkać się z~problemem \emph{wyścigów} (ang. \emph{Race Conditions}):
	dwa wątki w~tym samym czasie wykonują operacje na tych samych zasobach. Ponieważ system operacyjny przerywa pracę wątków w~sposób
	niezależny od działania poszczególnych wątków, może zdarzyć się sytuacja, gdy operacje jednego wątku \emph{przeplatają się} z~operacjami
	drugiego wątku (równie dobrze może to być o~wiele większa liczba wątków!) na tym samym zasobie. W~takiej sytuacji, wynik działania wątków
	może prowadzić do nieprzewidywalnych wyników, które mogą wypaczyć wynik działania aplikacji, lub nawet doprowadzić do przerwania jej wykonania.
\par
%
\indent
	Aby zaradzić temu problemowi, stworzony został mechanizm \emph{wzajemnego wykluczania} (ang. \emph{Mutual Exclusion, mutex}):
	jeśli dany zasób jest wolny, wówczas dostęp do niego otrzymuje dokładnie jeden wątek, zaś inne wątki czekają na zwolnienie tego zasobu.
	Wątki mogą być uśpione lub oczekiwać aktywnie na przydzielenie zasobu, jednakże są one w~jakiś sposób zablokowane.  
	Okazuje się, że mechanizm wzajemnego wykluczania wiąże się z~różnego rodzaju zagrożeniami. Najpoważniejsze z~nich to zakleszczenie 
	(ang. \emph{deadlock}) i~zagłodzenie (ang. \emph{starvation}).
\par
%
\indent
	Zakleszczenie występuje w~momencie gdy jeden wątek wykonuje operacje na zasobie A, a~następnie (nie zwalniając A) zgłasza zapotrzebowanie
	na zasób B, który jest zarezerwowany dla drugiego wątku, który oczekuje na zwolnienie zasobu A.
\par
%
\indent
	Zagłodzenie wątku występuje wtedy, gdy pewien wątek oczekuje na dostęp do zasobu, lecz nie jest on do tego zasobu dopuszczony.
\par

\subsection{Wątki przestrzeni użytkownika}
%
\indent
	Wątki przestrzeni użytkownika (ang. \emph{User--Level Threads, ULT}) nie są widoczne dla systemu operacyjnego w~taki sposób,
	jak ma to miejsce w~wypadku KLT. Podstawowa różnica pomiędzy tymi dwoma rodzajami wątków polega na podziale czasu pomiędzy
	poszczególne wątki: uruchomienie (wznawianie) oraz przerywanie działania KLT jest dokonywane przez system operacyjny, zaś 
	zarządzanie działaniem ULT jest zadaniem dla użytkownika. Strategie uruchamiania poszczególnych wątków przestrzeni jądra
	(ang. \emph{scheduling}) jest również odpowiedzialnością systemu operacyjnego, co w~przypadku ULT jest także powinnością 
	użytkownika. Wątki przestrzeni użytkownika są z~tego powodu o~wiele lżejsze: przełączanie pomiędzy nimi nie jest tak czasochłonne
	jak w~wypadku KLT. Co więcej, zarządzanie ULT może być bardziej dopasowane do charakteru aplikacji: niektóre wątki potrzebują
	prostego zarządzania (pierwszy wątek, drugi, \ldots, $n$-ty, pierwszy, \ldots -- algorytm \emph{Round-Robin}),
	inne aplikacje wymagają nadawania poszczególnym wątkom priorytetów, itp. Co więcej, przezroczystość ULT dla systemu nie zakłóca
	jego własnej polityki zarządzania wątkami. Model, który odpowiada wątkom przestrzeni użytkownika, to N:1.
	ULT są nazywane \emph{zielonymi wątkami} (ang. \emph{Green Threads}).
\par
%
\indent
	Zaletą tego rozwiązania jest także łatwość uniknięcia zjawiska wyścigów. Ponieważ wszystkie ULT są wykonywane
	w~ramach jednego procesu, wątek posiada wyłączny dostęp do wszystkich jego zasobów,
	nie modyfikowany przez żaden inny wątek.
\par
%
\indent
	Jeśli jeden z~wątków przestrzeni
	użytkownika zostanie zablokowany (np. podczas wykonywania operacji na IO), wówczas działanie innych wątków będzie również zablokowane.
	Aby uniknąć takich sytuacji, operacje blokujące są często opakowane w~taki sposób, by nie blokować wykonania wątku (\emph{jacketing}).
\par
%
\indent
	Ważną cechą ULT jest ich niezależność od systemu operacyjnego, w~jakim pracują: nie potrzebują one mechanizmu wywłaszczania
	(ang. \emph{preemption}), gdyż kontrola nad wykonaniem wątku jest wykonywana po stronie użytkownika.
\par
%
\subsection{Wątki mieszane}
%
\indent
	Wątki mieszane są implementacją modelu N:M. Wątki w~tym modelu są zarządzane zarówno 
	przez system operacyjny, jak też użytkownika. Ich celem jest połączenie zalet KLT i ULT. Istnieją implementacje, które
	modyfikują system operacyjny, jak również takie, które nie ingerują w~jego strukturę. Opisana w~tej pracy implementacja
	wątków mieszanych stosuje drugą metodę.
\par
\subsection{Wzajemne wykluczanie}
%
\indent
	Jak wspomniano wcześniej, wzajemne wykluczanie jest mechanizmem używanym w~celu uniknięcia zjawiska wyścigów.
	Aby wykluczyć taką sytuację, wątki wykonują działania na współdzielonym zasobie w~ramach tzw. \emph{sekcji krytycznej}
	(ang. \emph{critical section}).
\par
%
\indent
	Sekcję krytyczną można zapisać w~następujący sposób:
	\begin{verbatim}

    Wejście_do_sekcji_krytycznej();

    Sekcja_krytyczna(); // wykonanie operacji
                        // na współdzielonych zasobach

    Wyjście_z_sekcji_krytycznej();
	\end{verbatim}
	
	Operacje wejścia i~wyjścia z~sekcji krytycznej muszą spełniać następujące warunki:
	\begin{itemize}
	\item[{\bf M}] Wzajemne wykluczanie : dokładnie jeden wątek znajduje się w~sekcji krytycznej
	(Jeśli nastąpi przełączenie tego wątku na inny -- jest on nadal w~sekcji krytycznej, a~więc inne wątki
	w~dalszym ciągu muszą czekać na jego wyjście z~sekcji krytycznej).
	\item[{\bf P}] Postęp (ang. \emph{Progress}): jeśli żaden wątek nie znajduje się w~sekcji krytycznej, a~istnieją wątki oczekujące na wejście
	do własnych sekcji krytycznych, to decyzja o~wejściu do sekcji krytycznej zapada pomiędzy wątkami, które oczekują na wejście
	lub wychodzą z~sekcji krytycznej. Co więcej, decyzja ta musi zapaść (nie może być odraczana w~nieskończoność).
	\item[{\bf B}] Ograniczone czekanie (ang. \emph{Bounded waiting}): wątek znajdzie się w~sekcji krytycznej najpóźniej po określonej
	liczbie prób.
	\end{itemize}
\par
\subsubsection{Algorytm Petersona}
%
\indent
	Algorytm jest rozwiązaniem problemu sekcji krytycznej dla dwóch wątków. 
\par


%\end{document}
